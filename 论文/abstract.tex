\begin{abstract}
向量,又称矢量,是既有方向又有大小的量.向量在学术中,特别是在数学和物理中有着十分广泛的应用.高中阶段,我们学习了向量的数乘和向量积并借助向量工具解决立体几何问题.
	向量积,又称外积(与之对应的是内积,即数量积).向量积的概念在高中阶段很少提及.但,利用向量积可以快速而准确地解决某些(子)问题,并具有推广的潜质.
	本课题从一道简单而基础的数学题出发,引出一系列思考和讨论,探究向量在高中数学中的应用.本课题探究不全是教材中的向量,一些常规的向量方法将不再提及.

\end{abstract}