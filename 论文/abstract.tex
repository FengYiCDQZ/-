\begin{abstract}
向量,又称矢量,是既有大小又有方向的量.向量在学术,
特别是在数学和物理中有着十分广泛的应用.高中阶段,
我们学习了向量的数乘和数量积(内积),并借助向量工具
解决立体几何问题,例如我们利用巧妙的数量积进行角度的
运算.但是向量积(外积)的概念在高
中阶段很少提及.利用向量积可以快速而准确地解决
某些问题,并具有推广的潜质.本课题从
一道简单而基础的数学题出发,
展开一系列思考和讨论,讨论了平面中面积、空间中面积、
空间中体积、法向量等的求解方法,探究向量积在高中(高考)数学中的应用.
\end{abstract}