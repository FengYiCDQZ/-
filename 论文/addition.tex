\section{补充知识}

虽然这一部分叫\bm{补充知识},但是我们并没有准备在
这里做知识的新课式讲解.这里只会列出一些基本概念
和公式,供已经学习过但忘记的人复习用,如果您真的不知道
什么叫行列式,请自行购买线性代数学习.

\define $m\times n$个数排成每行$n$列的数表
$$\left(
  \begin{array}{cccc}
    a_{11} & a_{12} & \cdots & a_{1n} \\
    a_{21} & a_{22} & \cdots & a_{2n} \\
    \vdots & \vdots &  & \vdots \\
    a_{m1} & a_{m2} & \cdots & a_{mn} \\
  \end{array}
\right)$$
称为一个$m\times n$ \textbf{矩阵},简记为$\bm{A}=(a_{ij})$或$\bm{A}=(a_{ij})_{m\times n}$,$a_{ij}$称为矩阵的元素.

\define 一个$m\times m$的矩阵称为$\bm{m}$阶方阵

\define $n$个数$1,2,\cdots,n$排成一个有序$n$元数组称为一个 $\bm{n}$ \textbf{元排列}.

\define $S_n=$\{所有$n$元排列\}

\define 对于给定的$n$元排列$p_1,p_2,\cdots,p_n$,
如果一对数前后位置与它们的大小顺序相反(即左边的数大于右边的数),
则称它们为一个 \textbf{逆序}.
一个排列中逆序的总数称为该排列的 \textbf{逆序数},记作$$\tau(p_1,p_2,\cdots,p_n)$$.

\define 设$\bm{A}=(a_{ij})$,所谓$\bm{A}$\textbf{的行列式}或$\bm{n}$\textbf{阶行列式},是指由$A$确定的一个数,记为$\left | \bm{A} \right |$(或$\det \bm{A}$),这个数由下式来确定
$$
\left | \bm{A} \right |=\sum_{(i_1,\cdots,i_n)\in S_n}{(-1)^{\tau(i_1,\cdots,i_n)}a_{1i_1}a_{2i_2}\cdots a_{ni_n}}
$$

\define 在行列式
$$
D=
\left|
  \begin{array}{ccccc}
    a_{11} & \cdots & a_{1j} & \cdots & a_{1n} \\
    \vdots &  & \vdots &  & \vdots \\
    a_{i1} & \cdots & a_{ij} & \cdots & a_{in} \\
    \vdots &  & \vdots &  & \vdots \\
    a_{n1} & \cdots & a_{nj} & \cdots & a_{nn} \\
  \end{array}
\right|
$$
中划去$a_{ij}$所在的第$i$行和第$j$列,
留下的$(n-1)$阶行列式叫$(i,j)$元$a_{ij}$的余子式
,记作$M_{ij}$,记$$A_{ij}=(-1)^{i+j}M_{ij},$$
$A_{ij}$叫$(i,j)$元$a_{ij}$的代数余子式.

\define 不加证明地给出以下定理
$$
D=\sum_{j=1}^{n}a_{i_0j}A_{i_0j}
$$

\define 设$\bm{a},\bm{b}$是两个向量,规定$\bm{a}$与$\bm{b}$的向量积
是一个向量,记作$\bm{a}\times\bm{b}$,它的模与方向分别为\\
$(i) \abs{\bm{a}\times\bm{b}}=\abs{\bm{a}}\abs{\bm{b}}\sin<a,b>$\\
$(ii)\bm{a}\times\bm{b}\textbf{同时垂直于}\bm{a}\textbf{和}\bm{b},\textbf{并且}\bm{a},\bm{b},\bm{a}\times\bm{b}\textbf{符合右手法则}$\\
由定义得:\\
$\bm{i}\times\bm{j}=\bm{k},\bm{j}\times\bm{k}=\bm{i},\bm{k}\times\bm{i}=\bm{j}\\ 
\bm{a}\times\bm{a}=\bm{0}\\
\bm{a}\times\bm{b}=-\bm{b}\times\bm{a}
$


按照定义求向量的向量积不方便,现在我们考虑以下方法

\prob 已知$\bm{a}=(x_1,y_1,z_1),\bm{b}=(x_2,y_2,z_2)$,求$\bm{a}\times\bm{b}$\\
\sol
由定义知
$$\bm{a}=(x_1,y_1,z_1)
=x_1\bm{i}+y_1\bm{j}+z_1\bm{k},$$
$$\bm{b}=(x_2,y_2,z_2)=
x_2\bm{i}+y_2\bm{j}+z_2\bm{k},
$$
所以 
\begin{eqnarray}
  \bm{a}\times\bm{b}
  &=&(x_1\bm{i}+y_1\bm{j}+z_1\bm{k})\times(x_2\bm{i}+y_2\bm{j}+z_2\bm{k})\nonumber\\
  &=&x_1x_2\bm{i}\times\bm{i}+y_1y_2\bm{j}\times\bm{j}+z_1z_2\bm{k}\times\bm{k}+\nonumber\\
  &&x_1y_2\bm{i}\times\bm{j}+y_1z_2\bm{j}\times\bm{k}+z_2x_1\bm{k}\times\bm{i}+\nonumber\\
  &&x_2y_1\bm{j}\times\bm{i}+y_2z_1\bm{k}\times\bm{j}+z_2x_1\bm{i}\times\bm{k}\nonumber\\
  &=&\cdots\nonumber, 
\end{eqnarray}

最后我们有
$$
\bm{a}\times\bm{b}=
\left|
  \begin{array}{cc}
    y_1&z_1\\
    y_2&z_2\\
  \end{array}
\right|\bm{i}
+
\left|
  \begin{array}{cc}
    z_1&x_1\\
    z_2&x_2\\
  \end{array}
\right|\bm{j}
+
\left|
  \begin{array}{cc}
    x_1&y_1\\
    x_2&y_2\\
  \end{array}
\right|\bm{k},
$$
或
$$
\bm{a}\times\bm{b}=
\left|
  \begin{array}{ccc}
    \bm{i}&\bm{j}&\bm{k}\\
    x_1&y_1&z_1\\
    x_2&y_2&z_2\\
  \end{array}
\right|.
$$

更多有关行列式的知识可以阅读参考文献或
\href{https://www.zhihu.com/question/36966326}{知乎}