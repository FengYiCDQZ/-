\section{分析与总结}
向量的向量积有着广泛的用处
\begin{itemize}
  \item 面积
  \item 体积
  \item 法向量
\end{itemize}

\subsection{向量积在解析几何中的用处}
在解析几何中利用向量积可以快速而简洁地表示出面积,
有时会有意想不到的妙处.

在下一道例题中,标准解答(原题中椭圆已知)复杂而冗余,
但如果利用参数方程和向量积,则几乎没有什么计算量.

\prob 已知椭圆$\dps\frac{x^2}{a^2}+\frac{y^2}{b^2}=1(a>b>0)$
上有两动点$A$、$B$,求$S_{\triangle OAB}$最大值.

\sol设$$
    A(a\cos\alpha,b\sin\alpha),
    B(a\cos\beta,b\sin\beta),$$
则
\begin{eqnarray}
S_{\triangle OAB} 
&=&\frac{1}{2}\abs{\vv{OA}\times\vv{OB}}\nonumber\\
&=&\frac{1}{2}\abs{ab\sin\beta\cos\alpha-ab\sin\alpha\cos\beta}\nonumber\\
&=&\frac{1}{2}ab\abs{\sin(\alpha-\beta)}\nonumber\\
&\leq&\frac{1}{2}ab\nonumber.
\end{eqnarray}
\solend

\subsection{向量积在立体几何中的用处}

\prob 在空间直角坐标系中,已知$A(a,0,0),B(0,b,0),C(0,0,c)$,求面
$ABC$的法向量$\bm{n}$.

\sol 
\begin{eqnarray}
  \bm{n}&=&
  \left|
  \begin{array}{cccc}
    \bm{i}&\bm{j}&\bm{k}&1\\
    a&0&0&1\\
    0&b&0&1\\
    0&0&c&1
  \end{array}
  \right|\nonumber\\
  &=&(bc,ca,ab).\nonumber
\end{eqnarray}
\solend

式(\ref{mixedProduct})实际上给出了空间中四面体的体积公式,
利用它可以很快的表示出空间中的体积,由于高中阶段此类
题目较少,这里不再做说明.

在立体几何中,利用向量积可以快速地算出
体积、法向量和空间面积,
进而还可以算出点到直线距离,点到面距离,
减少多余的计算,节省时间,本文不再赘述.


\subsection{规律}

\bm{一维}$\quad$ 已知一维空间中一点$A(x_1)$,则线段$|OA|$的长度为
$$ 
\left|x_1\right|.
$$

\bm{二维}$\quad$ 已知二维空间中两点$A(x_1,y_1),B(x_2,y_2$,
则三角形$OAB$的面积为
$$ 
\frac{1}{2}\cdot
\left|
\begin{array}{cc}
  x_1&y_1\\
  x_2&y_2
\end{array}
\right|.
$$

\bm{三维}$\quad$ 已知三维空间中三点
$A(x_1,y_1,z_1),B(x_2,y_2,z_2),C(x_3,y_3,z_3)$,则四面体$OABC$的体积为
$$
\frac{1}{6}\cdot
\left|
\begin{array}{ccc}
  x_1&y_1&z_1\\
  x_2&y_2&z_2\\
  x_3&y_3&z_3
\end{array}
\right|. 
$$

\bm{$n$维}$\quad$ 已知$n$维空间中$n$点
$A_1(a_{11},a_{12},\cdots,a_{1n}),
A_2(a_{21},a_{22},\cdots,a_{2n}),
\cdots,A_n(a_{n1},a_{n2},\cdots,a_{nn})$
则\\``体''$A_1A_2\cdots A_n$的``体积''为
$$
\frac{1}{n!}\cdot
\left|
\begin{array}{cccc}
  a_{11}&a_{12}&\cdots&a_{1n}\\
  a_{21}&a_{22}&\cdots&a_{2n}\\
  \vdots&\vdots&      &\vdots\\
  a_{n1}&a_{n2}&\cdots&a_{nn}\\
\end{array}
\right|? 
$$

\subsection{其他}
由于向量积并未在高中阶段学习,超出考纲范围,
所以尽管有些时候使用向量积非常方便,
但是不推荐在解答题\footnote{文子龙说``可以扯把子$\cdots$''}的过程呈现中使用
向量积.但同时,在一些选择题和填空题中使用向量积
也无妨.

