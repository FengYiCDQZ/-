\part{分析与总结}
向量的向量积有着广泛的用处
\begin{itemize}
  \item 面积
  \item 体积
  \item 法向量
\end{itemize}

在解析几何中利用向量积可以快速而简洁地表示出面积,
有时会有意想不到的妙处,如下一道例题,标准解答(原题中椭圆已知)复杂而冗余,
但如果利用参数方程和向量积,则几乎没有什么计算量.


\prob 已知椭圆$\dps\frac{x^2}{a^2}+\frac{y^2}{b^2}=1(a>b>0)$上有两动点$A$、$B$,求$S_{\triangle OAB}$最大值\\
\from{方若愚}$\dps\textrm{设}A(a\cos\alpha,b\sin\alpha),B(a\cos\beta,b\sin\beta)\\
\so S_{\triangle OAB}=\frac{1}{2}\abs{\vv{OA}\times\vv{OB}}=\frac{1}{2}\abs{ab\sin\beta\cos\alpha-ab\sin\alpha\cos\beta}\\
=\frac{1}{2}ab\abs{\sin(\alpha-\beta)}\leq\frac{1}{2}ab$

在立体几何中,利用向量积可以快速地算出
体积、法向量和空间面积,
进而还可以算出点到直线距离,点到面距离,
减少多余的计算,节省时间.

